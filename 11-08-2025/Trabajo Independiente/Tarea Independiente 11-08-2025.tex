\documentclass[12pt]{article}
\usepackage[utf8]{inputenc}
\usepackage{graphicx}
\usepackage{hyperref}
\usepackage{geometry}
\usepackage{caption}
\usepackage{float}
\usepackage{enumitem}
\geometry{margin=1in}
\usepackage{listings}
\usepackage{xcolor}

% Colores personalizados
\definecolor{keywordcolor}{RGB}{0,0,180}
\definecolor{ndkeywordcolor}{RGB}{128,0,128}
\definecolor{stringcolor}{RGB}{163,21,21}
\definecolor{commentcolor}{RGB}{0,128,0}

% --- Definición de Java ---
\lstdefinelanguage{Java}{
	keywords={abstract, assert, boolean, break, byte, case, catch, char, class, const, continue, default, do, double, else, enum, extends, final, finally, float, for, goto, if, implements, import, instanceof, int, interface, long, native, new, package, private, protected, public, return, short, static, strictfp, super, switch, synchronized, this, throw, throws, transient, try, void, volatile, while},
	keywordstyle=\color{keywordcolor}\bfseries,
	comment=[l]{//},
	morecomment=[s]{/*}{*/},
	commentstyle=\color{commentcolor}\ttfamily,
	stringstyle=\color{stringcolor}\ttfamily,
	morestring=[b]",
	morestring=[b]'
}

% --- Definición de JavaScript ---
\lstdefinelanguage{JavaScript}{
	keywords={break, case, catch, continue, debugger, default, delete, do, else, finally, for, function, if, in, instanceof,
		new, return, switch, this, throw, try, typeof, var, void, while, with, const, let},
	keywordstyle=\color{keywordcolor}\bfseries,
	ndkeywords={class, export, boolean, throw, implements, import, this},
	ndkeywordstyle=\color{ndkeywordcolor}\bfseries,
	identifierstyle=\color{black},
	sensitive=false,
	comment=[l]{//},
	morecomment=[s]{/*}{*/},
	commentstyle=\color{commentcolor}\ttfamily,
	stringstyle=\color{stringcolor}\ttfamily,
	morestring=[b]',
	morestring=[b]"
}

% Configuración global
\lstset{
	basicstyle=\ttfamily\small,
	numbers=left,
	numberstyle=\tiny\color{gray},
	stepnumber=1,
	numbersep=5pt,
	showstringspaces=false,
	tabsize=4,
	breaklines=true,
	captionpos=b,
	frame=single,
	frameround=tttt
}

\title{Tarea Independiente 11/08/2025}
\author{David Núñez Franco}
\date{\today}

\begin{document}
	
	\maketitle
	
	\section*{Inventario de Conceptos Claves}
	
	\begin{itemize}
		\item Lambda y el cálculo lambda de Church
		\item Tipos en tiempo de compilación
		\item Tipos en tiempo de ejecución
		\item Lambdas en Java
		\item Método versus lambda en Java
		\item El rol del Typer en Java 
		\item Lenguaje estáticamente tipado versus dinámicamente tipado
		\item AST (equivalente al árbol de expresión de Estructuras Discretas)
		\item Tablas de símbolos
		\item Typer de Java
		\item lambda versus método en Java 
		\item Tipo genérico
		\item Método como parámetro de tipo genérico
		\item Extra: preprocesador, macros y templates en C++
	\end{itemize}
	
	\section*{Ejercicio 1}
	
	\begin{quote}
		a) Considere esta función add en JS:
		
		\begin{lstlisting}[language=JavaScript]
			const add = (f, g) => x => f(x) + g(x)
		\end{lstlisting}

		Pruebe que da lo mismo add(add(f, g), h) que add(f, add(g, h)) es decir, add es asociativa.
		
		b) Construya una lambda zero tal que, add() = 0,  add(zero, f) dé igual que add(f, zero) y dé igual que f para cualquier lambda f.
		
		c) Construya una función max\_function(f, g) que cumpla: max\_function(f,g)(x) = max(f(x), g(x)) para todo x de tipo number. dode max(x, y) calcula el máximo entre cualesquiera números x y y.
	\end{quote}
	
	\subsection*{Solución punto 1.a}
	
	Ejercicio demostrado en el cuaderno de práctica.
	
	\subsection*{Solución punto 1.b}
	
	Dicho problema pide construir una función zero que sea el elemento neutro de la operación add. Debe cumplir que para cualquier funcion f:
	
	\begin{itemize}
		\item add(zero, f) = f
		\item add(f, zero) = f	
	\end{itemize}
	
	Además, add(zero, zero) produce siempre 0 + 0 = 0
	
	\begin{lstlisting}[language=JavaScript,caption={Solucion en JavaScript}]
		// DEFINIMOS ADD COMO LAMBDA
		const add = (f, g) => (x) => f(x) + g(x);
		
		// DEFINIMOS ZERO COMO LAMBDA. RETORNA CERO
		const zero = (x) => 0;
		
		// EJEMPLO DE FUNCIONES
		const f = (x) => 2 * x;
		const g = (x) => x + 1;
		
		// PROBAMOS
		console.log(add(zero, f)(5)); // resultado es 10 = 2 * 5
		console.log(add(f, zero)(5)); // resultado es 10 = 2 * 5
		console.log(add(zero, zero)(5)); // resultado es 0
	\end{lstlisting}
	
	\subsection*{Solución punto 1.c}
	
	La función max\_function devuelve el mayor de los números que recibe. Luego, se aplica a los valores calculados por f, g para el mismo x. El resuktado será una nueva función que, dado un número, siempre entrega el mayor de f(x) y g(x).
	
	\begin{lstlisting}[language=JavaScript, caption={Solución en JavaScript}]
		// max_function retorna una funcion que calcula el maximo de f(x) y g(x)
		const max_function_with_math = (f, g) => (x) => Math.max(f(x), g(x));
		const max_function_without_math = (f, g) => (x) => f(x) >= g(x) ? f(x) : g(x);
		
		// construimos algunas funciones
		const f = (x) => 2 * x;
		const g = (x) => x + 5;
		
		// probamos
		console.log(max_function_with_math(f, g)(2)); // resultado es 7, Math.max(4,7)
		console.log(max_function_with_math(f, g)(10)); // resultado es 20, Math.max(20,15)
	\end{lstlisting}
	
	\section*{Ejercicio 2}
	
	\begin{quote}
		Haga lo equivalente que 1 pero en Java.
	\end{quote}
	
	\subsection*{Solución punto 2.b}
	
	\begin{lstlisting}[language=Java,caption={Solución en Java}]
		import java.util.function.Function;
		
		public class Ejercicio2B
		{
			// add: suma de funciones
			public static Function<Integer, Integer> add(Function<Integer, Integer> f, Function<Integer, Integer> g) {
				return x -> f.apply(x) + g.apply(x);
			}
			
			// zero: elemento neutro
			public static Function<Integer, Integer> zero = x -> 0;
			
			public static void main(String[] args) {
				// creamos las funciones
				Function<Integer, Integer> f = x -> 2 * x;
				Function<Integer, Integer> g = x -> x + 1;
				
				// prueba
				System.out.println(add(zero, f).apply(5)); // 10
				System.out.println(add(f, zero).apply(5)); // 10
				System.out.println(add(zero, zero).apply(5)); // 0
			}
		}
	\end{lstlisting}
	
	\subsection*{Solución punto 2.c}
	
	\begin{lstlisting}[language=Java,caption={Solución en Java}]
		import java.util.function.Function;
		
		public class Ejercicio2C {
			// max_function utilizando Math.max
			public static Function<Integer, Integer> maxFunctionWithMath(Function<Integer, Integer> f, Function<Integer, Integer> g) {
				return x -> Math.max(f.apply(x), g.apply(x));
			}
			
			// max_function sin Math.max
			public static Function<Integer, Integer> maxFunctionWithoutMath(Function<Integer, Integer> f, Function<Integer, Integer> g) {
				return x -> f.apply(x) >= g.apply(x) ? f.apply(x) : g.apply(x);
			}
			
			public static void main(String[] args) {
				// construimos las funciones
				Function<Integer, Integer> f = x -> 2 * x;
				Function<Integer, Integer> g = x -> x + 5;
				
				// pruebas maxFunctionWithMath
				System.out.println(maxFunctionWithMath(f, g).apply(2));   // 7
				System.out.println(maxFunctionWithMath(f, g).apply(10));  // 20
				
				// prueba maxFunctionWithoutMath
				System.out.println(maxFunctionWithoutMath(f, g).apply(2));   // 7
				System.out.println(maxFunctionWithoutMath(f, g).apply(10));  // 20
			}
		}
	\end{lstlisting}
		
	\section*{Ejercicio 3}
	
	\begin{quote}
		Traduzca a Java el siguiente código 
		
		\begin{lstlisting}[language=Java]
			template <typename T> T max(T a, T b) {
				return (a > b) ? a : b;
			}
			int main() {
				int x = 5, y = 10;
				std::cout << max(x, y) << std::endl;
				double a = 666.5, b = 665.8;
				std::cout << max(a, b) << std::endl;
				return 0;
			}
		\end{lstlisting}

	\end{quote}
	
	\subsection*{Solución}
	
	Primero, debemos entender el problema. Utilizaremos genéricos con restricciones.
	
	\texttt{<T extends Comparable<? super T>> } exige que el tipo T implemente Comparable, accediendo al método compareTo para establecer un orden.
	
	\texttt{a.compareTo(b)} devuelve
	\begin{itemize}
		\item valor positivo si a mayor que b
		\item 0 si son iguales
		\item valor negativo si a menor que b
	\end{itemize}
	
	\begin{lstlisting}[language=Java, caption={Solución en Java, utilizando genéricos con restricciones}]
		public class main {
			/*T extends Comparable<? super T> permite que T use compareTo, para compararse con su tipo o super-tipos.*/
			public static <T extends Comparable<? super T>> T max(T a, T b) {
				/*compareTo devuelve:
				* 1. Un numero positivo si a > b
				* 2. Un cero si a == b
				* 3. Un numero negativo si a < b*/
				return (a.compareTo(b) > 0) ? a : b;
			}
			
			public static void main(String[] args) {
				Integer x = 5, y = 10;
				System.out.println(max(x, y));
				
				Double a = 666.5, b = 665.8;
				System.out.println(max(a, b));
			}
		}
	\end{lstlisting}
	
	\section*{Ejercicio 4}
	
	\begin{quote}
		Investigue los términos aplicables a lenguajes de programación  "fuertemente tipado" versus "débilmente tipado". Esté en capacidad de explicarlo con ejemplos.
	\end{quote}
	
	\subsection*{Fuertemente Tipado vs Débilmente Tipado}
	
	\begin{itemize}
		\item \textbf{Fuertemente tipado}: El lenguaje no permite usar valores de tipos incompatibles, sin conversión explícita. Las operaciones inválidas entre tipos fallan en compilación o en tiempo de ejecución.
		\item \textbf{Débilmente tipado}: El lenguaje aplica conversiones implícitas (coerción) incluso entre tipos distintos, lo que puede dar resultados inesperados.
	\end{itemize}
	
	\subsection*{Ejemplos}
	
	\subsubsection*{Fuertemente tipado}
	
	\begin{lstlisting}[language=Java,caption={Java}]
		int x = 5;
		String y = "10";
		int z = x + y; // ERROR DE COMPILACION
	\end{lstlisting}
	
	\begin{lstlisting}[language=Java,caption={Python, tipado dinámico pero fuerte}]
		5 + "10" # TypeError: unsupported operand type(s)
	\end{lstlisting}
	
	\subsubsection*{Débilmente tipado}
	
	\begin{lstlisting}[language=JavaScript,caption={JavaScript}]
		5 + "10" // "510" (concatena)
		"5" - 1 // 4 (convierte "5" a numero)
		[] == 0 // true 
	\end{lstlisting}
	
	\subsubsection*{Relación con tipado estático/dinámico}
	
	\begin{itemize}
		\item \textbf{Tipado fuerte/débil}: Se refiere a cuánta coerción implícita permite el lenguaje.
		\item \textbf{Tipado estático/dinámico}: Se refiere a si el tipo se verifica en compilación o en ejecución.
	\end{itemize}
	
	\begin{table}[h]
		\centering
		\begin{tabular}{|l|l|l|}
			\hline
			\textbf{Lenguaje} & \textbf{Tipado} & \textbf{Comprobación} \\ \hline
			Java       & Fuerte & Estático \\ \hline
			Python     & Fuerte & Dinámico \\ \hline
			JavaScript & Débil  & Dinámico \\ \hline
			C          & Débil  & Estático \\ \hline
		\end{tabular}
		\caption{Comparación de tipado y comprobación en distintos lenguajes}
		\label{tab:tipado}
	\end{table}
	
	
\end{document}
