\documentclass[12pt]{article}
\usepackage[utf8]{inputenc}
\usepackage{graphicx}
\usepackage{hyperref}
\usepackage{geometry}
\usepackage{caption}
\usepackage{float}
\usepackage{enumitem}
\geometry{margin=1in}
\usepackage{listings}
\usepackage{xcolor}

\lstdefinelanguage{JavaScript}{
	keywords={break, case, catch, continue, debugger, default, delete, do, else, finally, for, function, if, in, instanceof,
		new, return, switch, this, throw, try, typeof, var, void, while, with, const, let},
	keywordstyle=\color{blue}\bfseries,
	ndkeywords={class, export, boolean, throw, implements, import, this},
	ndkeywordstyle=\color{purple}\bfseries,
	identifierstyle=\color{black},
	sensitive=false,
	comment=[l]{//},
	morecomment=[s]{/*}{*/},
	commentstyle=\color{gray}\ttfamily,
	stringstyle=\color{red}\ttfamily,
	morestring=[b]',
	morestring=[b]"
}

\lstset{
	language=Java,
	basicstyle=\ttfamily\small,
	keywordstyle=\color{blue},
	commentstyle=\color{gray},
	stringstyle=\color{red},
	numbers=left,
	numberstyle=\tiny\color{gray},
	stepnumber=1,
	numbersep=5pt,
	showstringspaces=false,
	tabsize=4,
	breaklines=true,
	captionpos=b
}

\title{Tarea Independiente 04/08/2025}
\author{David Núñez Franco}
\date{\today}

\begin{document}
	
	\maketitle
	
	\section*{Inventario de Conceptos Claves}
	
	\begin{itemize}
		\item Rol de Arquitecto versus Ingeniero en el diagrama (realidad-persona-modelo-implementación)
		\item UML y rol en la historia del modelado
		\item En compiladores:
		\begin{itemize}
			\item Syntax y semántica (estática) (de nuevo)
			\item AST (Abstract Syntax Tree)
			\item Tabla de símbolos (symbol table)
		\end{itemize}
		\item En java
		\begin{itemize}
			\item Semántica de package
			\item Nombre simple
			\item Nombre completo (calificado)
			\item Mapeo de package a ruta en disco
			\item Semántica de import
			\item Semántica de import static
			\item Semántica del * en el import
			\item Semántica modificador de visibilidad (4 P's)
			\begin{itemize}
				\item public
				\item private
				\item protected
				\item default
			\end{itemize}
		\end{itemize}
		\item Principio Don't Repeat Yourself (DRY)
		\item var en JAVA
	\end{itemize}
	
	\section*{Punto 1}
	
	\begin{quote}
		Verifique con un ejemplo que javac detecta una posible ambigüedad en un identificador no calificado al querer compilar un archivo Test.java.
	\end{quote}
	
	\subsection*{Concepto}
	Antes de comenzar con la implementación, debemos definir el concepto de ambigüedad. En java, una ambigüedad en un identificador no calificado ocurre cuando una variable, clase o método puede referirse a más de una cosa, y el compilador no puede decidir cuál utilizar.
	
	\subsection*{Solución}
	
	\begin{lstlisting}[caption={A.java}]
		package paquete1;
		
		public class A {
			public static int value = 10;
		}
	\end{lstlisting} 
	
	\begin{lstlisting}[caption={B.java}]
		package paquete2;
		
		public class B {
			public static int value = 10;
		}
	\end{lstlisting} 
	
	\begin{lstlisting}[caption={Test.java}]
		import static paquete1.A.value;
		import static paquete2.B.value;
		
		public class Test {
			public static void main(String[] args) {
				System.out.println(value); // Aqui se genera la ambiguedad
			}
		}
	\end{lstlisting}
	
	\subsection*{Prueba y error}
	
	\begin{verbatim}
		mkdir -p paquete1 paquete2
		
		mv A.java paquete1/
		mv B.java paquete2/
		
		javac paquete1/A.java paquete2/B.java Test.java
	\end{verbatim}
	
	\begin{figure}[H]
		\centering
		\includegraphics[width=0.95\textwidth]{error.png}
		\caption{Mensaje error ambigüedad}
		\label{fig:01}
	\end{figure}
	
	\section*{Punto 2}
	
	\begin{quote}
		¿Por qué javac no permite que dentro de un método que se declare un identificador público o privado?
		
		\begin{lstlisting}
			class A{
				public int foo(){
					public int x = 666;
					return x;
				}
			}
		\end{lstlisting}
	\end{quote}
	
	\subsection*{Solución}
	
	En Java, los modificadores de acceso (public, private, protected) se utilizan para controlar la visibilidad de clases, métodos y atributos a nivel de clase o instancia. Dentro de un método, solo se pueden declarar variables locales, y estas:
	
	\begin{itemize}
		\item Solo existen mientras se ejecuta el método.
		\item No forman parte de la clase ni del objeto.
		\item No pueden tener modificadores de acceso, porque no tiene sentido controlar su visibilidad fuera del método (nadie más puede acceder a ellas de todos modos).
	\end{itemize}
	
	Por eso, cuando tenemos
	
	\begin{lstlisting}
		class A{
			public int foo(){
				public int x = 666;
				return x;
			}
		}
	\end{lstlisting}
	
	javac lanza error, porque public no está permitido en la declaración de una variable local dentro del método foo.
	
	\section*{Punto 3}
	
	\begin{quote}
		 Considere el código siguiente dentro de una clase Test.java.  Note que hay unos ??? que indican un lugar donde Ud. debe escribir, según se explica. Solo puede cambiar esos ??? :
		 
		 \begin{lstlisting}
		 	class A {
		 		private int x = 666;
		 		public class B {
		 			private int x = 999;
		 			public void foo() {
		 				System.out.println(???);
		 			}
		 		}
		 	}
		 	public class Test{
		 		public static void main(String... args){
		 			A a = new A();
		 			var b = ??? // Se necesita crear un b tal que
		 			b.foo();    // b.foo() imprima 666 no 999
		 		}
		 	}
		 \end{lstlisting} 
		 
		 Solo se vale cambiar los ??? por algo correcto para que se logre imprimir el valor del x de A y no el del B
	\end{quote}
	
	\subsection*{Solución}
	
	\begin{lstlisting}
		class A {
			private int x = 666;
			
			public class B {
				private int x = 999;
				
				public void foo() {
					System.out.println(A.this.x); // accedemos al x de A
				}
			}
		}
		
		public class Test {
			public static void main(String... args) {
				A a = new A();
				var b = a.new B(); // instanciamos la clase interna B con el objeto a
				b.foo();           // imprime 666
			}
		}
	\end{lstlisting}
	
	\subsection*{Ejecución y compilación}
	
	\begin{verbatim}
		javac Test.java
		java Test
	\end{verbatim}
	
	\begin{figure}[H]
		\centering
		\includegraphics[width=0.95\textwidth]{salida.png}
		\caption{Salida}
		\label{fig:02}
	\end{figure}
	
\end{document}
