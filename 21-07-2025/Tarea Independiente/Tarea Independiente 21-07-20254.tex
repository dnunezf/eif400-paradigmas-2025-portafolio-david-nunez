\documentclass[12pt]{article}
\usepackage[utf8]{inputenc}
\usepackage{graphicx}
\usepackage{hyperref}
\usepackage{geometry}
\usepackage{caption}
\usepackage{float}
\usepackage{enumitem}
\geometry{margin=1in}

\title{Tarea Independiente 21/07/2025\\Prueba de Prompt con \texttt{ollama3} y el modelo \texttt{gemma3}}
\author{David Núñez Franco}
\date{\today}

\begin{document}
	
	\maketitle
	
	\section*{Resumen}
	Como introducción al modelo gemma3 sobre la plataforma ollama3, se verificó si el modelo \texttt{gemma3} ejecutado con \texttt{Ollama} hace uso de una GPU en un sistema Linux. Luego, se realizó una prueba experimental utilizando el modelo \texttt{gemma3} sobre la plataforma \texttt{ollama3}. El objetivo fue ejecutar un prompt creativo y humorístico centrado en la cultura costarricense y el Club Sport Herediano, seguido de un análisis crítico sobre IA generativa como paradigma de programación.
	
	\section*{Verificación de uso de GPU en Ollama con el modelo \texttt{gemma3}}
	
	\subsection*{Ejecución del modelo con \texttt{--verbose}}
	
	Se corrió el modelo gemma3 usando el modo detallado (\texttt{--verbose}) para observar el tiempo de respuesta y posibles indicios del uso de GPU:
	
	\begin{figure}[H]
		\centering
		\includegraphics[width=0.95\textwidth]{verbose.png}
		\caption{Ejecución con --verbose}
		\label{fig:01}
	\end{figure}
	
	\subsection*{Verificación del tipo de procesador en uso}
	
	Para confirmar si el modelo estaba corriendo sobre CPU o GPU, se utilizó el comando \texttt{ollama ps}, que muestra el estado de ejecución actual del modelo:
	
	\begin{figure}[H]
		\centering
		\includegraphics[width=0.95\textwidth]{ps.png}
		\caption{Verificación con ollama ps}
		\label{fig:02}
	\end{figure}
	
	La salida indica que el modelo está usando \textbf{100\% CPU}, lo cual confirma que no se está utilizando GPU.
	
	\subsection*{Compatibilidad del modelo con mi máquina}
	
	Se investigó el soporte de GPU por parte de Ollama:
	
	\begin{itemize}
		\item Ollama detecta automáticamente GPU compatibles (NVIDIA o Apple Silicon).
		\item En este caso, mi equipo es un IdeaPad-5-15ITL05 con gráficos Intel integrados, los cuales actualmente no son compatibles con Ollama para ejecución en GPU.
		\item No se cuenta con una tarjeta gráfica NVIDIA, por lo que no se puede usar nvidia-smi para confirmar la carga de GPU.
	\end{itemize}
	
	\section*{Prompt Utilizado (Parte 2: Humor)}
	
	\begin{quote}
		Eres un robot recién activado en Costa Rica, específicamente en la ciudad de Heredia. Al despertar, lo primero que ves es una camisa del Club Sport Herediano, y decides que tu propósito en la vida es convertirte en delantero estrella del "Team Florense". 
		
		Describe tu primer entrenamiento con el equipo: cómo tratas de usar tus piernas para hacer chilenas, confundes el balón con una olla arrocera, y en algún momento cantás el himno con acento argentino sin razón aparente. 
		
		Sé dramático, patriótico y exagerado. Mezcla términos de fútbol con referencias a Costa Rica: gallo pinto, "pura vida", "diay", etc. Termina con una declaración heroica al estilo novelero: ".... mi corazón es rojo y amarillo!".
	\end{quote}
	
	\section*{Resultado (Parte 2)}
	
	\begin{figure}[H]
		\centering
		\includegraphics[width=0.95\textwidth]{hallazgo.png}
		\caption{Respuesta humorística por parte del modelo}
		\label{fig:03}
	\end{figure}
	
	\section*{Prompt Utilizado (Parte 3: Análisis técnico con DeepSeek)}
	
	\begin{quote}
		Reflexiona críticamente sobre si la inteligencia artificial generativa, especialmente modelos como ChatGPT o Codex, puede considerarse un nuevo paradigma de programación, más allá de los paradigmas tradicionales como la programación orientada a objetos (OOP), la programación funcional (FP) o la programación lógica (LP).
		
		En tu análisis, considera aspectos como:
		\begin{itemize}
			\item ¿Qué define un paradigma de programación?
			\item ¿En qué se diferencia el uso de IA generativa frente a escribir código manualmente?
			\item ¿La IA generativa representa una nueva forma de relacionarse con el software?
			\item ¿Podría cambiar la manera en que enseñamos programación?
		\end{itemize}
	\end{quote}
	
	\section*{Resultado y Reflexión(Parte 3: Análisis)}
	
	\subsection*{Definición de paradigma de programación}
	Un paradigma de programación es un modelo o enfoque que define cómo se estructura y ejecuta el código. Los paradigmas tradicionales incluyen:
	
	\begin{itemize}
		\item \textbf{Programación imperativa (procedural)}: Instrucciones secuenciales (ej: C).
		\item \textbf{Orientación a objetos (OOP)}: Modelado con clases y objetos (ej: Java).
		\item \textbf{Programación funcional (FP)}: Funciones puras y sin estado mutable (ej: Haskell).
		\item \textbf{Programación lógica (LP)}: Reglas y consultas basadas en lógica (ej: Prolog).
	\end{itemize}
	
	Cada paradigma impone una forma distinta de pensar y resolver problemas.
	
	\subsection*{¿La IA generativa (ChatGPT, Codex) es un nuevo paradigma?}
	La IA generativa no encaja directamente en los paradigmas tradicionales, pero podría considerarse un meta-paradigma o paradigma asistido por IA, ya que:
	
	\begin{itemize}
		\item No reemplaza los paradigmas existentes, sino que actúa como una capa de abstracción superior.
		\item Cambia la relación humano-código; en lugar de escribir manualmente, se describe el problema en lenguaje natural y la IA sugiere soluciones.
	\end{itemize}
	
	Ejemplo:
	
	\begin{itemize}
		\item \textbf{Tradicional (OOP)}: Diseñar una clase Persona con atributos y métodos manualmente.
		\item \textbf{Con IA}: Prompt: \textit{"Genera una clase Persona en Python con nombre, edad y método para saludar."}
	\end{itemize}
	
	\subsection*{¿Nueva forma de relacionarse con el software?}
	
	\begin{itemize}
		\item \textbf{Ventajas}:
		\begin{itemize}
			\item \textbf{Reduce barrera de entrada}: Personas sin experiencia pueden prototipar código.
			\item \textbf{ Acelera el desarrollo}: Tareas repetitivas (ej: CRUDs, boilerplate) se generan automáticamente.
			\item \textbf{Cambia el rol del programador}: Menos tiempo en sintaxis, más en diseño y validación.
		\end{itemize}
		\item \textbf{Limitaciones}:
		\begin{itemize}
			\item La IA puede generar código incorrecto o inseguro.
			\item Sin entender el código generado, se dificulta la depuración.
			\item No reemplaza el razonamiento algorítmico.
		\end{itemize}
	\end{itemize}
	
	\subsection*{¿Impacto en la enseñanza de programación?}
	\begin{itemize}
		\item \textbf{Oportunidades}:
		\begin{itemize}
			\item Menos memorización, más pensamiento lógico.
			\item Aprender a formular prompts efectivos y validar código generado.
		\end{itemize}
		\item \textbf{Desafíos}:
		\begin{itemize}
			\item Riesgo de que estudiantes no dominen fundamentos.
			\item Necesidad de enseñar \textit{debugging} de código generado.
		\end{itemize}
	\end{itemize}
	
\end{document}
